\pagebreak
\section{Experiment-2}
\textbf{Q.1 What would you expect the steady-state velocity of the system to be if a step input of 2000
N were applied at t = 0?}
\\
Ans. \\
From newtons second law we have, dv/dt = (F-bv)/M.
\\ M is the mass, v is velocity and F is force, b is damping cofficient. 
\\
At steady state, the change in velocity will be zero. i.e. dv/dt = 0.
\\
Therefore, F = bv
\\F = 2000N
\\b = 40 N*sec/m
\\2000 = 40*v
\\v= 50 m/sec.

\textbf{Q.2 Why do this step input and the ramp input with saturation we simulated have the same steady state velocity?}
\\Ans.The step input is similar to saturated ramp input, because when steady state is reached, both step input and saturated ramp input
have fixed value(input value). Therefore, if we apply same force to same system (under idential condition), 
then system will attain the same velocity. That is why we get same steady state velocity.

\textbf{Q.3 Using which input, the step or the ramp with saturation, would you expect the system to reach its steady-state velocity more quickly?
}
\\Ans. With step input the system will reach it's steady state velocity more quickly. Because, 
saturated ramp input will take more time to reach it's saturation value whereas step input 
reaches to it's fixed value quickly. 